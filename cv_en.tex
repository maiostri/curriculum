%%%%%%%%%%%%%%%%%%%%%%%%%%%%%%%%%%%%%%%%%%%%%%%%%%%%%%%%%%%%%%%%%%%%%%%%
%%%%%%%%%%%%%%%%%%%%%% Simple LaTeX CV Template %%%%%%%%%%%%%%%%%%%%%%%%
%%%%%%%%%%%%%%%%%%%%%%%%%%%%%%%%%%%%%%%%%%%%%%%%%%%%%%%%%%%%%%%%%%%%%%%%

%%%%%%%%%%%%%%%%%%%%%%%%%%%%%%%%%%%%%%%%%%%%%%%%%%%%%%%%%%%%%%%%%%%%%%%%
%% NOTE: If you find that it says                                     %%
%%                                                                    %%
%%                           1 of ??                                  %%
%%                                                                    %%
%% at the bottom of your first page, this means that the AUX file     %%
%% was not available when you ran LaTeX on this source. Simply RERUN  %%
%% LaTeX to get the ``??'' replaced with the number of the last page  %%
%% of the document. The AUX file will be generated on the first run   %%
%% of LaTeX and used on the second run to fill in all of the          %%
%% references.                                                        %%
%%%%%%%%%%%%%%%%%%%%%%%%%%%%%%%%%%%%%%%%%%%%%%%%%%%%%%%%%%%%%%%%%%%%%%%%

%%%%%%%%%%%%%%%%%%%%%%%%%%%% Document Setup %%%%%%%%%%%%%%%%%%%%%%%%%%%%

% Don't like 10pt? Try 11pt or 12pt
\documentclass[10pt]{article}

% This is a helpful package that puts math inside length specifications
\usepackage{calc}

%lingua e tal
\usepackage[brazil]{babel}
\usepackage[latin1]{inputenc}


% Simpler bibsection for CV sections
% (thanks to natbib for inspiration)
\makeatletter
\newlength{\bibhang}
\setlength{\bibhang}{1em}
\newlength{\bibsep}
 {\@listi \global\bibsep\itemsep \global\advance\bibsep by\parsep}
\newenvironment{bibsection}
    {\minipage[t]{\linewidth}\list{}{%
        \setlength{\leftmargin}{\bibhang}%
        \setlength{\itemindent}{-\leftmargin}%
        \setlength{\itemsep}{\bibsep}%
        \setlength{\parsep}{\z@}%
        }}
    {\endlist\endminipage}
\makeatother

% Layout: Puts the section titles on left side of page
\reversemarginpar

\usepackage[paper=letterpaper,
            %includefoot, % Uncomment to put page number above margin
            marginparwidth=1.2in,     % Length of section titles
            marginparsep=.05in,       % Space between titles and text
            margin=1in,               % 1 inch margins
            includemp]{geometry}

%% Use these lines for A4-sized paper
%\usepackage[paper=a4paper,
%            %includefoot, % Uncomment to put page number above margin
%            marginparwidth=30.5mm,    % Length of section titles
%            marginparsep=1.5mm,       % Space between titles and text
%            margin=25mm,              % 25mm margins
%            includemp]{geometry}

%% More layout: Get rid of indenting throughout entire document
\setlength{\parindent}{0in}

%% This gives us fun enumeration environments. compactitem will be nice.
\usepackage{paralist}

%% Reference the last page in the page number
%
% NOTE: comment the +LP line and uncomment the -LP line to have page
%       numbers without the ``of ##'' last page reference)
%
% NOTE: uncomment the \pagestyle{empty} line to get rid of all page
%       numbers (make sure includefoot is commented out above)
%
\usepackage{fancyhdr,lastpage}
\pagestyle{fancy}
%\pagestyle{empty}      % Uncomment this to get rid of page numbers
\fancyhf{}\renewcommand{\headrulewidth}{0pt}
\fancyfootoffset{\marginparsep+\marginparwidth}
\newlength{\footpageshift}
\setlength{\footpageshift}
          {0.5\textwidth+0.5\marginparsep+0.5\marginparwidth-2in}
\lfoot{\hspace{\footpageshift}%
       \parbox{4in}{\, \hfill %
                    \arabic{page} of \protect\pageref*{LastPage} % +LP
%                    \arabic{page}                               % -LP
                    \hfill \,}}

% Finally, give us PDF bookmarks
\usepackage{color,hyperref}
\definecolor{darkblue}{rgb}{0.0,0.0,0.3}
\hypersetup{colorlinks,breaklinks,
            linkcolor=darkblue,urlcolor=darkblue,
            anchorcolor=darkblue,citecolor=darkblue}

%%%%%%%%%%%%%%%%%%%%%%%% End Document Setup %%%%%%%%%%%%%%%%%%%%%%%%%%%%


%%%%%%%%%%%%%%%%%%%%%%%%%%% Helper Commands %%%%%%%%%%%%%%%%%%%%%%%%%%%%

% The title (name) with a horizontal rule under it
%
% Usage: \makeheading{name}
%
% Place at top of document. It should be the first thing.
\newcommand{\makeheading}[1]%
        {\hspace*{-\marginparsep minus \marginparwidth}%
         \begin{minipage}[t]{\textwidth+\marginparwidth+\marginparsep}%
                {\large \bfseries #1}\\[-0.15\baselineskip]%
                 \rule{\columnwidth}{1pt}%
         \end{minipage}}

\renewcommand{\section}[2]%
        {\pagebreak[2]\vspace{1.3\baselineskip}%
         \phantomsection\addcontentsline{toc}{section}{#1}%
         \hspace{0in}%
         \marginpar{
         \raggedright \scshape #1}#2}

% An itemize-style list with lots of space between items
\newenvironment{outerlist}[1][\enskip\textbullet]%
        {\begin{itemize}[#1]}{\end{itemize}%
         \vspace{-1em}}

\newenvironment{compactouterlist}[1][\enskip\textbullet]%
        {\begin{itemize} \setlength{\itemsep}{0.1em}}{\end{itemize}%
         \vspace{-1em}}

% An environment IDENTICAL to outerlist that has better pre-list spacing
% when used as the first thing in a \section
\newenvironment{lonelist}[1][\enskip\textbullet]%
        {\vspace{-\baselineskip}\begin{list}{#1}{%
        \setlength{\partopsep}{0pt}%
        \setlength{\topsep}{0pt}}}
        {\end{list}\vspace{-.6\baselineskip}}

% An itemize-style list with little space between items
\newenvironment{innerlist}[1][\enskip\textbullet]%
        {\begin{compactitem}[#1]}{\end{compactitem}}

% An environment IDENTICAL to innerlist that has better pre-list spacing
% when used as the first thing in a \section
\newenvironment{loneinnerlist}[1][\enskip\textbullet]%
        {\vspace{-\baselineskip}\begin{compactitem}[#1]}
        {\end{compactitem}\vspace{-.6\baselineskip}}

% To add some paragraph space between lines.
% This also tells LaTeX to preferably break a page on one of these gaps
% if there is a needed pagebreak nearby.
\newcommand{\blankline}{\quad\pagebreak[2]}

% Uses hyperref to link DOI
\newcommand\doilink[1]{\href{http://dx.doi.org/#1}{#1}}
\newcommand\doi[1]{doi:\doilink{#1}}



%%%%%%%%%%%%%%%%%%%%%%%% End Helper Commands %%%%%%%%%%%%%%%%%%%%%%%%%%%

%%%%%%%%%%%%%%%%%%%%%%%%% Begin CV Document %%%%%%%%%%%%%%%%%%%%%%%%%%%%

\begin{document}
\makeheading{Ricardo Pereira Maiostri}


\section{Personal information}
%
% NOTE: Mind where the & separators and \\ breaks are in the following
%       table.
%
% ALSO: \rcollength is the width of the right column of the table
%       (adjust it to your liking; default is 1.85in).
%
\newlength{\rcollength}\setlength{\rcollength}{1.85in}%
%
\begin{tabular}[t]{@{}p{\textwidth-\rcollength}p{\rcollength}}
\textit{Phone:} (19) 9198-4687 & \textit{City:} Santa Barbara Do Oeste-SP \\
\textit{E-mail:} \href{mailto:maiostri@gmail.com}{maiostri@gmail.com} & \textit{Marital status:} Single \\
\textit{github:} \href{http://github.com/maiostri}{http://github.com/maiostri} &  \textit{Birthday:} 19/09/1988 \\

\end{tabular}


\section{Education}
\href{http://www.usp.br/}{\textbf{University of Sao Paulo}}, USP.


\begin{outerlist}
\item[] Computer Science (B.S) Degree, \hfill\textbf{January 2008 - August 2013}       
\end{outerlist}

\blankline

\href{http://www.colegiopolitec.com.br/}{\textbf{Technical School Politec}},

\begin{outerlist}
\item[] Software Applications and Programming, \hfill\textbf{2004-2006}       
\end{outerlist}

\blankline

\section{Experience}
%

\href{http://www.venturus.org/}{\textbf{Instituto Venturus}},
Campinas - SP
\begin{outerlist}

\item[] \textit{Software Developer Intern}%
    \hfill \textbf{Since july 2012}
    \begin{innerlist}
        \item Participation in the planning and execution of an ASP(Advanced Schedule Planning) product, first for one of the institute clients(an eletronic chinese corporation) and after for the industry. 
        Worked in back-end development, using Java EE and technologies such as \textit{Restful webservices}, \textit{JSON}, \textit{Hibernate/JPA}, \textit{Drools}, \textit{Jbehave} and \textit{Maven}, as on the front-end development, using 
         \textit{Javascript}, \textit{JQuery} and \textit{JQuery UI}. \\
        \item Participation in the planning of one of the courses of the educational division of the institute, the \textit{Venturus Education}. The course \textit{Game development for mobile plataforms} deals with the complete chain of development for a game in the plataforms Android and iOS, using the \textit{Lua} language and the API Corona. \\
    \end{innerlist}    

\end{outerlist}

\href{http://www.gbdi.icmc.usp.br/}{\textbf{GBDI - Database and Image research group - USP}},
S�o Carlos - SP
\begin{outerlist}

\item[] \textit{Undergrad researcher - CNPQ}%
    \hfill \textbf{Since july 2011}
    \begin{innerlist}
        \item Worked in \textit{Siren}, a framework whose goal is provide extensions
        for the SQL language in a way to support similiraty queries in complex data, such as images, audio and geo-positioning data. Acting first on making framework portable between multiple operating systems, 
        and after in the planning and development of new modules and features for the framework.
        \textit{Siren} supports \textit{Oracle} and \textit{PostgreSQL}. The changes were made from the C++ Builder language to ANSI C++, using the libraries \textit{boost} and \textit{stl}.
    \end{innerlist}    
\end{outerlist}

\blankline

\href{http://portal.tododia.uol.com.br/}{\textbf{TodoDia - Newspaper}},
Americana - SP
\begin{outerlist}

\item[] \textit{Software Developer Intern}%
    \hfill \textbf{July 2006 until January 2007}
    \begin{innerlist}
        \item Planning and development of various systems for the newspaper, such as classifiels and newspaper sign, using Delphi 7 with the databases MaxDB and Firebird. 
    \end{innerlist}    
\end{outerlist}

\blankline

\href{http://www.darquia.com}{\textbf{Darquia - Software development office}},
Santa B�rbara Do Oeste - SP
\begin{outerlist}

\item[] \textit{Software Developer Intern}%
    \hfill \textbf{January 2006 until June 2006}
    \begin{innerlist}
        \item Development of comercial software for the automobilistic and textile areas, using Delphi 7. 
    \end{innerlist}
    
\end{outerlist}

\blankline

\section{Competences}
%
\textbf{Languages}
\begin{compactouterlist}
\item[] English.
\item[] Basic French.
\end{compactouterlist}

\blankline

\textbf{Programming languages}
\begin{compactouterlist}
\item[] Java, Javascript, C++, C, Scala, Delphi.
\end{compactouterlist} 

\blankline

\textbf{Web technologies}
\begin{compactouterlist}
\item[] JQuery, JQuery UI, Ajax, CSS, XHTML.
\end{compactouterlist}

\blankline

\textbf{Mobile plataforms}
\begin{compactouterlist}
\item[] Android.
\end{compactouterlist}

\blankline

\textbf{Databases}
\begin{compactouterlist}
\item[] Oracle, PostgreSQL, Firebird.
\end{compactouterlist}

\blankline

\textbf{Operating Systems}
\begin{compactouterlist}
\item[] Windows, Linux, BSD.
\end{compactouterlist}

\end{document}

%%%%%%%%%%%%%%%%%%%%%%%%% End CV Document %%%%%%%%%%%%%%%%%%%%%%%%%%%%%
